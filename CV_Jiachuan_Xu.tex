%
% LaTeX source of my resume
% =========================
%
% Heavily commented to to fit even LaTeX beginners (hopefully).
%
% See the `README.md` file for more info.
%
% This file is licensed under the CC-NC-ND Creative Commons license.
%


% Start a document with the here given default font size and paper size.
\documentclass[10pt,a4paper]{article}

% Set the page margins.
\usepackage[a4paper,margin=0.75in]{geometry}

% Setup the language.
\usepackage[english]{babel}
\hyphenation{Some-long-word}
\usepackage{sectsty}
%\sectionfont{\rmfamily\mdseries\Large}
%\subsectionfont{\rmfamily\mdseries\itshape\large}

% Makes resume-specific commands available.
\usepackage{resume}

%\usepackage[T1]{fontenc}
%\usepackage[urw-garamond]{mathdesign}

\usepackage{graphicx}
\usepackage{bm}


%%%%%%%%%%%%%%%%%%{}%%%%%%%%%%%%%%%%%%%
%for small dot before item in itemize
\renewcommand{\labelitemi}{$\vcenter{\hbox{\tiny$\bullet$}}$}
\newcommand{\RNum}[1]{\uppercase\expandafter{\romannumeral #1\relax}}

\begin{document}  % begin the content of the document
\sloppy  % this to relax whitespacing in favour of straight margins
\maintitle{}{}{Last update on \today}

\begin{minipage}[t]{0.495\textwidth}
{\huge Jiachuan Xu}
\end{minipage}
% Alternatively, print name centered and bold:
%\centerline{\huge \bf \name}
\begin{minipage}[t]{0.495\textwidth}
    \includegraphics[totalheight=0.8in]{thu-whole-logo.png}
\end{minipage}
\bigskip

\begin{minipage}[t]{0.495\textwidth}
  Department of Physics \\
  Tsinghua University \\
  Haidian District\\
  Beijing, P.R.China
  %to be added
\end{minipage}
\begin{minipage}[t]{0.495\textwidth}
  Phone: (+86) 131-4134-5463 \\
  \phantom{Phone: }(+1) 626-550-6765\\
  Email: \href{mailto:xjc14@tsinghua.org.cn}{xjc14@tsinghua.org.cn}\\
  \phantom{Email:} \href{mailto:jiachuanxu14@gmail.com}{jiachuanxu14@gmail.com}\\
  Github: \href{https://github.com/JiachuanXu}{https://github.com/JiachuanXu}
\end{minipage}

% title on top of the document


\nobreakvspace{0.3em}  % add some page break averse vertical spacing

% \noindent prevents paragraph's first lines from indenting
% \mbox is used to obfuscate the email address
% \sbull is a spaced bullet
% \href well..
% \\ breaks the line into a new paragraph


\roottitle{EDUCATION}

\spacedhrule{0.4em}{0.8em}  % a horizontal line with some vertical spacing before and after

\headedsection
  {Tsinghua University}
  {\textsc{Beijing, China}}
  {
    \headedsubsection
    {B.S. in Physics}
    {2014.8 -- 2018.7}
    {}

    \headedsubsection
    {\textnormal{Cumulative GPA:} 87.49/100}
    %\textnormal{, Ranking: }13/52 \textnormal{in the Department of Physics}}
    {}
    {}
  }


\roottitle{RESEARCH EXPERIENCE}

\spacedhrule{0.4em}{0.8em}

\headedsection
  {Tsinghua University\\
  Department of Physics, \href{http://astro.tsinghua.edu.cn/}{Tsinghua Center for Astrophysics}}
  {\textsc{Beijing, China}} 
  {

    \headedsubsection
    {\textnormal{Research Assistant, Advisor:} Professor Yi Mao}
    {2016.6 -- Present}
    {
      \bodytext{
      \textbf{Project: Redshift Space Distortion (RSD) of the 21-cm Background From The Epoch of Reionization}
      \begin{itemize}
        \item Developed a method, $\tau$-MMRRM, to correct for the RSD in the 21-cm brightness temperature in numerical simulations.
        \item Developed a C program to generate distorted 21-cm brightness temperature in redshift space, and a toolbox to analyze the statistics of observables, e.g. power spectrum, probability distribution function(PDF), etc. The code is available on my github repository (\href{https://github.com/JiachuanXu/MMRRM_adv.git}{https://github.com/JiachuanXu/MMRRM\_adv.git}) %\footnote{Currently this repository is a private one. I'll change the status into public once I submit the application, but with a preliminary version of code.}
        \item Proposed the ``extended quasi-linear scheme" to interpret the 21-cm power spectrum in redshift space; quantified to what extent we can recover the matter power spectrum from 21-cm power spectrum. 
      \end{itemize}
      }
    }
  }

\headedsection
  {University of Arizona\\
  Department of Astronomy}
  {\textsc{Tucson, AZ, U.S.A}}
  {
    \headedsubsection
    {\textnormal{Research Assistant, Advisor:} Professor Xiaohui Fan, Professor Zheng Cai}
    {2018.1 -- Present}
    {
      \bodytext{
      \textbf{Project: Studying the Overdensity of Lyman Break Galaxies in Proto-cluster BOSS1441}
      \begin{itemize}
        \item Reduced optical and infrared images within BOSS1441 field, which contains one of the most massive proto-clusters at z=2.32.
        \item Combining photometry in UViJHK bands to deduce photometric redshift, quantifying the overdensity of Lyman Break Galaxies in BOSS1441.
        \item Reduced Binospec multislit spectra within BOSS1441 field.
      \end{itemize}
      }
    }
  }

\roottitle{Publications in Preparation (First Author)}
\spacedhrule{0.4em}{0.8em}
\begin{itemize}
    \item \textbf{Jiachuan Xu}, Yi Mao, "Redshift Space Distortion of the 21-cm Background from the Epoch of Reionization \RNum{2}: Effect of Finite Optical Thickness", in preparation. Expected submission to MNRAS at March, 2019%\footnote{Will put the link of the draft at here once it's finished.}
    \item \textbf{Jiachuan Xu}, Yi Mao, "Redshift Space Distortion of the 21-cm Background from the Epoch of Reionization \RNum{3}: Understanding RSD Through Extended Quasi-linear Scheme", in preparation. Expected submission to MNRAS at March, 2019%\footnote{Will put the link of the draft at here once it's finished.}
\end{itemize}

\roottitle{Publications (Co-author)}
\spacedhrule{0.4em}{0.8em}
\begin{itemize}
    \item Kai Hoffmann, Yi Mao, \textbf{Jiachuan Xu}, Houjun Mo, Benjamin D. Wandelt, "Signatures of Cosmic Reionization on the 21cm 2- and 3-point Correlation Function I: Quadratic Bias Modeling", 2018, \href{https://arxiv.org/abs/1802.02578}{arXiv:1802.02578}, submitted to MNRAS.
    \item F. Arrigoni Battaia, Chian-Chou Chen, M. Fumagalli, Zheng Cai, G. Calistro Rivera, \textbf{Jiachuan Xu}, I. Smail, J. X. Prochaska, Yujin Yang, C. De Breuck, "Overdensity of submillimeter galaxies around the z$\sim$2.3 MAMMOTH-1 nebula", 2018,  \href{https://arxiv.org/abs/1810.10140}{arXiv:1810.10140}, submitted to Astronomy \& Astrophysics.
\end{itemize}

\roottitle{Conference and Meeting}

\spacedhrule{0.4em}{0.8em}

\headedsection
    {American Astronomy Society}
    {\textsc{Seattle, United States}}
    {
        \headedsubsection
        {223rd AAS Winter Meeting}
        {2019.1(Expected)}
        {
          \bodytext{
            \begin{itemize}
            \item Give poster on "Redshift Space Distortion of the 21-cm Background from the Epoch of Reionization"
            \end{itemize}
          }
        }
    }

\headedsection
  {School On Cosmology, Fudan University}
  {\textsc{Shanghai, China}}
  {
    \headedsubsection
    {2017 Spring School On Cosmology: "Early Universe: Theory and Observations"}
    {2017.2}
    {    
      \bodytext{
      \begin{itemize}
        \item Accomplished courses on large-scale structure and early universe.
      \end{itemize}
      }
    }
  }

\headedsection
  {China Astronomical Society}
  {}
  {
    \headedsubsection
    {The $19^{th}$ CAS Guoshoujing Symposium on Galaxies and Cosmology}
    {2016.6}
    {
    \bodytext{
      \begin{itemize}
        \item Learned the frontiers in galaxies and cosmology.
      \end{itemize}
      }
    }

    \headedsubsection
    {2017 Annual Meeting}
    {2017.8}
    {
    \bodytext{
      \begin{itemize}
        \item Learned the frontiers in cosmology, large-scale structure and galaxies formation.
      \end{itemize}
      }
    }
  }

%\headedsection
%  {21cm Cosmology and SKA Workshop and Tianlai Collaboration Meeting}
%  {2017.8, Urumqi}
%  {
%  \bodytext{
%    \begin{itemize}
%      \item Learned the scientific goal of Tianlai Project, and instrumental issues.
%    \end{itemize}
%  }
%  }

\roottitle{SKILLS}

\spacedhrule{0.4em}{0.8em}

\headedsection
  {Language Proficiency:}
  {}
  {
  \begin{itemize}
    \item Fluent in English
    \item TOEFL: 103(Reading 27, Listening 28, Speaking 22, Writing 26)
  \end{itemize}
  }

\headedsection
  {Programming \& Data Reduction:}
  {}
  {
  \begin{itemize}
    \item Skilled: C, Python3, \LaTeX
    \item Familiar: Mathematica, Fortran, IDL, C++
    \item Library \& Software: \href{https://www.openmp.org/}{OpenMP}, \href{http://www.fftw.org/}{FFTW}, \href{https://www.astromatic.net/software/sextractor}{SExtractor}, \href{https://github.com/gbrammer/eazy-photoz.git}{EAZY}, \href{http://www.cfht.hawaii.edu/~arnouts/LEPHARE/lephare.html}{LePhare}...
  \end{itemize}  
  }

\roottitle{COURSEWORK}

\spacedhrule{0.4em}{0.8em}

\headedsection
  {}
  {}  
  {
  \bodytext{  
  \begin{minipage}[t]{0.495\textwidth}
  \begin{itemize}
  \setlength{\itemsep}{1pt}
  \setlength{\parsep}{2pt}
  \setlength{\parskip}{2pt}
    \item General Relativity 94
    \item Statistical Mechanics(1) 92
    \item Astrophysics 93
    \item Quantum Mechanics(1) 97
  \end{itemize}
  \end{minipage}
  \begin{minipage}[t]{0.495\textwidth}
  \begin{itemize}
  \setlength{\itemsep}{1pt}
  \setlength{\parsep}{2pt}
  \setlength{\parskip}{2pt}
    \item Differential Geometry 90
    \item Advanced Observational Astrophysics 95
    \item Group Theory 95
    \item Nuclear and Particle Physics 91
  \end{itemize}
  \end{minipage}  
  }
  }

\roottitle{HONORS \& AWARDS}

\spacedhrule{0.4em}{0.8em}

\headedsection
  {\textbf{2017}\phantom{1} Scholarship of Outstanding Social Work}
  {}
  {}

\headedsection
  {\textbf{2015}\phantom{1} Scholarship of Outstanding Voluntary Public Service}
  {}
  {}

\roottitle{SOCIAL SERVICES \& Outreach}

\spacedhrule{0.4em}{0.8em}

\headedsection
  {Team leader of Department of Physics Volunteer Association.}
  {2016.9 -- 2017.9}
  {
    \bodytext{
      \begin{itemize}
        \item Initiated the voluntary activity "The Amazing Physics D.I.Y.": Volunteers in physics department guided kids to do interesting physics experiments.
      \end{itemize}
    }
  }

\end{document}